\chapter{Synopsis of results}
\label{chap:d:tng}\vspace{-2em}

Using the \dnmtchip and other hypomorphic \proteinnamemouse{Dnmt1} mouse models, the former PhD student Lena Vockentanz had demonstrated the importance of DNA methylation for self-renewal in leukemia stem cells (LSCs)\cite{Vockentanz2011}.  However, it remained unresolved, which gene programs were affected and if other mechanisms such as senescence or chromatin instability would be involved, too.  

This doctoral project was part of a joint effort with Irina Savelyeva, a postdoc in our laboratory, to identify epigenetically regulated genes and mechanisms with high relevance for leukemia self-renewal. The respective objectives of the project were the bioinformatic analysis and interpretation of comprehensive massive parallel sequencing datasets as well as the experimental validation of the findings in vitro:

\begin{itemize}
	\item For the first time, whole-genome bisulfite sequencing was carried out for a mouse model of \dnmtwt as well as \dnmtchip \mllafnine leukemia. An in-depth characterization of the methylomes was performed.
	\item A novel method for methylome analyses based on generalized additive models was developed. The new technique allowed to accurately quantify the methylation persistency across large regions and simultaneously to account for distinctive deviations within spatially constrained regions such as CpG-Islands. 
	\item To assess the consequences of the \dnmtchip genotype, sophisticated bioinformatic procedures, such as reference-guided transcriptome assembly or analysis of Hi-C data, were applied.
	\item Since anomalous enhancers emerged as important factors in leukemogenesis, bivalently transcribed active enhancers were called and comprehensively characterized, including motifs and methylation status.
	\item Numerous third-party datasets were integrated with our own results to put them into context and to pinpoint the most relevant enhancers for \mllafnine leukemia.
	\item More than \num{110000} lines of code were written for all analyses combined.  
	\item Although these results are not presented herein, selected genes and enhancers were experimentally tested in vitro by shRNA knock-down or CRISPRi for their effect on self-renewal and growth rate.
\end{itemize}
