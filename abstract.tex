\begin{abstract}
	
Over the past four decades, little progress was made in the standard therapy of acute myeloid leukemia (AML). Only in recent years, the standard anthracycline/cytarabine combination chemotherapy is increasingly replaced by or supplemented with new substances such as gemtuzumab, ozogamicin, enasidenib or midostaurin. None the less, an unmet need for specific drugs stands, since AML is a heterogeneous disease and many subtypes still lack molecularly targeted therapy options. 

To aid the identification of new, specific molecular therapy targets, we utilized a mouse model to elicit acute myeloid leukemia in a DNA hypomethylation background. We proposed that silenced tumor-suppressor genes would become reactivated due to insufficient methylation and the model could thus point us to new molecular targets. 

Indeed, we observed that the model required appropriate DNA methylation levels to maintain transcriptional sanity, to avoid senescence and to ultimately preserve its full self-renewal capability. Therefore, we developed a new analysis method for methylation data to elaborate on the underlying causes and consequences. We showed that, contrarily to our initial proposal, no reactivation of genes by promoter hypomethylation occurred. Subsequently, we explored possible alternatives to explain the phenotype.    

Since misplaced or anomalous enhancers have emerged as important contributing factors of leukemogenesis, we asked whether enhancers might be sites of therapeutically relevant DNA methylation changes. Here we present a comprehensive characterization of bivalently transcribed active enhancers and their respective methylation status. Our analysis highlights a GC-rich subgroup of regulatory elements, which are unmethylated on DNA level, but characterized by high \hisfourthree in leukemia as well as in various regular hematopoietic lineages. These elements resembled bivalently marked promoters, were presumably bound by \proteinnamemouse{Mll2/COMPASS} and targeted by the histone demethylase \proteinnamemouse{Utx}(\genenamemouse{Kdm6a}) for activation. Hence, it is suggested that specific \proteinnamemouse{Utx} inhibitors upon availability should be investigated with regard to their therapeutic potential in \proteinnamemouse{Mll}-rearranged leukemia. 

%\begin{keywords}
% 	Cancer, Leukemia, MLL-AF9, Epigenetics, DNA-methyltransferase, Enhancers, Transcriptional noise, 
% \end{keywords}

\end{abstract}