\chapter{Epigenetics} 
\minitoc
\section{Epigenetics}
\label{chap:i:epigenetics:overview}

Even before the structure of DNA was resolved in 1953\cite{Watson1953} or its role as the carrier of hereditary information was proven in 1944\cite{Avery1944,McCarty1946}, developmental biologists in the 1930s had made substantial progress towards the understanding of embryonic development. Despite lacking knowledge about the biochemical background of these effects, they produced models, which linked genes, mutations and other factors to phenotypical outputs. Therefore, the initial concept of epigenetics referred to the influence of additional factors besides the gene\footnote{as illustrated by the greek prefix $\epsilon\phi\iota$ meaning \emph{in addition to} or \emph{on top of}} that would shape the phenotype of an organism. However, soon thereafter the focus 

histone modifications \cite{Allfrey1964}
chromatin states\cite{Ernst2010}

epigenetic memory\cite{DUrso2014}
nucleic acid modifications \cite{Chen2016a}
nucleic acid motifications on RNA \cite{Wang2014,Roundtree2017}

Epigenetics: core misconcept \cite{Ptashne2013}

%%%%%% wörtlich aus Srivastava %%%%
The carboxy-terminal domain (CTD) of RNAPII in eukaryotes undergoes extensive posttranslational modification, called the 'CTD code' that is indispensable for coupling transcription with many cellular processes, including mRNA processing. The posttranslational modification of histones, known as the 'histone code', is also critical for gene transcription through the reversible and dynamic remodeling of chromatin structure. Notably, the histone code is closely linked with the CTD code, and their combinatorial effects enable the delicate regulation of gene transcription. \cite{Srivastava2015}
%%%%%%%%%%%%%

\section{DNA methylation}

The occurrence of methylated cytosine was parenthetically mentioned by Erwin Chargaff\cite{Chargaff1951} and proven \cite{Cohn1951}

DNA methylation was shown to be involved in important functions like transcriptional silencing of genes or endogenous retroviruses, imprinting, X chromosome inactivation and is relevant for genomic stability\cite{Goll2005}. At large it occurs in the context of CG dinucleotides (CpG) and its maintenance is mainly ensured by DNA (cytosine-5)-methyltransferase 1 (Dnmt1), which preferentially acts on hemimethylated DNA. Dnmt1 activity is most pronounced near DNA replication sites during S phase, but also independently detectable in M and G2 phases. Impaired activity, by inhibition or reduced Dnmt1 expression, results in a gradual hypomethylation of DNA and may interfere with aforementioned processes. For our study we used the Dnmt1 -/chip mouse strain\cite{Gaudet2003}, which carries a null\cite{Lei1996} and a hypomorphic\cite{Tucker1996} allele and exhibits $\leq$ 30\% Dnmt1 expression. 

Such DNA modifications include canonical 5-methylcytosine (5mC), 5-hydroxymethylcytosine (5hmC), 5-formylcytosine (5fC) and 5-carboxycytosine (5caC)\cite{Plongthongkum2014}

%%%%%%%%%%%%% wörtlich aus Uniprot %%%%%%%%%%%%%
Dioxygenase that catalyzes the conversion of the modified genomic base 5-methylcytosine (5mC) into 5-hydroxymethylcytosine (5hmC) and plays a key role in active DNA demethylation. Also mediates subsequent conversion of 5hmC into 5-formylcytosine (5fC), and conversion of 5fC to 5-carboxylcytosine (5caC). Conversion of 5mC into 5hmC, 5fC and 5caC probably constitutes the first step in cytosine demethylation. Methylation at the C5 position of cytosine bases is an epigenetic modification of the mammalian genome which plays an important role in transcriptional regulation. In addition to its role in DNA demethylation, plays a more general role in chromatin regulation. Preferentially binds to CpG-rich sequences at promoters of both transcriptionally active and Polycomb-repressed genes. Involved in the recruitment of the O-GlcNAc transferase OGT to CpG-rich transcription start sites of active genes, thereby promoting histone H2B GlcNAcylation by OGT. Also involved in transcription repression of a subset of genes through recruitment of transcriptional repressors to promoters. Involved in the balance between pluripotency and lineage commitment of cells it plays a role in embryonic stem cells maintenance and inner cell mass cell specification. Plays an important role in the tumorigenicity of glioblastoma cells. TET1-mediated production of 5hmC acts as a recruitment signal for the CHTOP-methylosome complex to selective sites on the chromosome, where it methylates H4R3 and activates the transcription of genes involved in glioblastomagenesis (PubMed:25284789).6 Publications
Caution
Subsequent steps in cytosine demethylation are subject to discussion. According to a first model cytosine demethylation occurs through deamination of 5hmC into 5-hydroxymethyluracil (5hmU) and subsequent replacement by unmethylated cytosine by the base excision repair system (PubMed:21496894). According to another model, cytosine demethylation is rather mediated via conversion of 5hmC into 5fC and 5caC, followed by excision by TDG and replacement by unmethylated cytosine.1 Publication
%%%%%%%%%%%%%

Rates of somatic mutations and loss of CpGs\cite{Behjati2014}
Dnmt3A and 3B in HSC\cite{Challen2014}
Review Establishing, maintaining and modifying DNA methylation patterns in plants and animals \cite{Law2010}
Reprogramming the methylome during development\cite{Lee2014}


Alternative Route how DNA methylation is actively removed by DNMT3A and 3B\citerev{Wijst2015}
Interestingly, mammalian DNA methyltransferases 3A and 3B (DNMT-3A and -3B) have also been reported to induce active DNA demethylation, in addition to their well-known function in catalyzing methylation. In situations of extremely low levels of S-adenosyl methionine (SAM), DNMT-3A and -3B might demethylate C-5 methyl cytosine (5mC) via deamination to thymine, which is subsequently replaced by an unmodified cytosine through the base excision repair (BER) pathway. Alternatively, 5mC when converted to 5- hydroxymethylcytosine (5hmC) by TET enzymes, might be further modified to an unmodified cytosine by DNMT-3A and -3B under oxidized redox conditions, although exact pathways are yet to be elucidated. I


cis-effect will lead to thalassaemia, the beta globin gene is hypermethylated and inactive\cite{Kioussis1983}

interplay with histone modifications \cite{Putiri2014}

TET1, a member of a novel protein family, is fused to MLL in acute myeloid leukemia containing the t (10; 11)(q22; q23) \cite{Lorsbach2003}

DNA-methylation-related genes were altered in 44\% of all samples, which had at least one nonsynonymous mutation\cite{TCGAC2013}, 

DNA methylation and interplay with H3K27 modification at enhancers and promoters\cite{King2016a}

TFs bind methylation dependent\cite{Yin2017a}

DNA methylation and mutational burden in colon cancer\cite{Poulos2017}

applicability of circulating DNA from blood plasma for cancer detection\cite{Shen2018}

Eukaryotic genomes contain numerous non-functional high-affinity sequences for transcription factors. These sequences potentially serve as natural decoys that sequester transcription factors. We have previously shown that the presence of sequences similar to the target sequence could substantially impede association of the transcription factor Egr1 with its targets. In this study, using a stopped-flow fluorescence method, we examined the kinetic impact of DNA methylation of decoys on the search process of the Egr1 zinc-finger protein. We analyzed its association with an unmethylated target site on fluorescence-labeled DNA in the presence of competitor DNA duplexes, including Egr1 decoys. DNA methylation of decoys alone did not affect target search kinetics. In the presence of the MeCP2 methyl-CpG-binding domain (MBD), however, DNA methylation of decoys substantially (1030fold) accelerated the target search process of the Egr1 zinc-finger protein. This acceleration did not occur when the target was also methylated. These results suggest that when decoys are methylated, MBD proteins can block them and thereby allow Egr1 to avoid sequestration in non-functional locations. This effect may occur in vivo for DNA methylation outside CpG islands (CGIs) and could facilitate localization of some transcription factors within regulatory CGIs, where DNA methylation is rare.\cite{Kemme2017}


That this variability is associated with coherent, genome-scale, oscillations in DNA methylation with an amplitude dependent on CpG density.These observations provide fresh insights into the emergence of epigenetic heterogeneity during early embryo development, indicating that dynamic changes in DNA methylation might influence early cell fate decisions\cite{Rulands2018}


Importantly, although \proteinnamemouse{Dnmt1} is the only canonical methyltransferase exhibiting a replication foci targeting sequence (RFTS) necessary for its targeting to replication foci\citerev{Lyko2018} and is the major maintenance methyltransferase, it does have a limited de novo methyltransferase activity, too\cite{Cai2017}.

\section{Measuring DNA methylation}

sci-MET - Singe cell DNA methylation profiling\cite{Mulqueen2018}


% % % % % % % % % % wörtlich aus Wreczycka % % % % % % % % % % % % %
DNA methylation is one of the main epigenetic modifications in the eukaryotic genome; it has been shown to play a role in cell-type specific regulation of gene expression, and therefore cell-type identity. Bisulfite sequencing is the gold-standard for measuring methylation over the genomes of interest. Here, we review several techniques used for the analysis of high-throughput bisulfite sequencing. We introduce specialized short-read alignment techniques as well as pre/post-alignment quality check methods to ensure data quality. Furthermore, we discuss subsequent analysis steps after alignment. We introduce various differential methylation methods and compare their performance using simulated and real bisulfite sequencing datasets. We also discuss the methods used to segment methylomes in order to pinpoint regulatory regions. We introduce annotation methods that can be used for further classification of regions returned by segmentation and differential methylation methods. Finally, we review software packages that implement strategies to efficiently deal with large bisulfite sequencing datasets locally and we discuss online analysis workflows that do not require any prior programming skills. The analysis strategies described in this review will guide researchers at any level to the best practices of bisulfite sequencing analysis.\cite{Wreczycka2017}

\section{c/chip mouse model}