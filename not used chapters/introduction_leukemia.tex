\chapter{Leukemia and hematopoietic malignancies} 
\minitoc
\label{chap:i:leukemia:introduction}
% \SI{97}{\percent}


single cell, single variant mapping in human hematopoiesis\cite{Ulirsch2019}

Over the past four decades little has changed in the standard therapy of acute myeloid leukemia (AML), which is still mostly treated with a anthracycline/cytarabine combination chemotherapy\cite{Cohen1977} and has a dismal survival prognosis. Although clinical trials addressing the unmet need for specific drugs are ongoing\cite{beatAML}, AML is a heterogeneous disease and many subtypes still lack molecularly targeted therapy options. 

EU market approval for Decitabine in AML treatment\cite{Nieto2016}	

AML with nucleophosmin 1 NP1 mutations as founder mutation\cite{Cocciardi2019}

To aid the understanding, we have employed a mouse model, 
Hypomethylation of DNA Analogous to decitabine, which blocks DNA methyltransferases irreversibly, 



Decitabine irreversibly binds to DNA methyltransferases, which copy DNA methylation across cell divisions or create it de novo. 

The Dnmt1 -/chip mouse model, which 

Among the sequenced human cancer genomes, those of acute myeloid leukemia are characterized by remarkably few recurrent mutations in the coding regions of genes\cite{Kandoth2013,TCGAC2013}. The few genes affected are often epigenetic regulators\cite{Plass2013,Shen2013} and the resulting epigenetic abnormalities are believed to constitute strongly to the cancerogenesis\cite{Wen2018}. This especially holds true for the prevalent AML subtype in early childhood\cite{Schafer2010}. 

The latter is exemplified by the histone methyltransferase \proteinnamehuman{Mll1} \cite{Ziemin-vanderPoel1991,Thirman1993}, whose locus 11q23 is faulted in roughly 40\% of the infant (0-3 years of age) cases of acute myeloid leukemia, but seldom affected in older patients bearing more preleukemic mutations\cite{Bolouri2017}. A partial duplication inside \genenamehuman{Kmt2a}, the gene corresponding to the \proteinnamehuman{Mll1} protein, is sufficient to elicit leukemia\cite{Basecke2006}, although chromosomal rearrangements are far more common\cite{Li2014a}. To date, more than 60 fusion partners\cite{VanderBurg1999,Slany2016} and multiple breakpoints\cite{Kobayashi1993} have been reported for the \genenamehuman{Kmt2a} gene, however only one rearrangement is frequent enough to be recognized as separate entity in the WHO classification of
myeloid neoplasms and acute leukemia\cite{Arber2016}: The prevailing fusion occurs to the \proteinname{mixed-lineage leukemia translocated 3} gene \genenamehuman{Mllt3}\cite{Rowley1973}, whose protein product \proteinnamehuman{AF9} is a component of the super elongation complex (SEC)\cite{He2010,Lin2010}. The pathogenic fusion protein is termed \proteinnamehuman{MLL-AF9} and can be found in 9.5\% of childhood and in 0.5\% of adult -mostly therapy-related\cite{Super1993}- cases of acute myeloid leukemia\cite{Balgobind2009}.

Human \proteinnamehuman{MLL-AF9} can also transform mouse cells and educe a well characterized leukemia model system\cite{Somervaille2009,Krivtsov2013,Stavropoulou2016} out of several early hematopoietic lineages\cite{George2016}. Importantly, it is known that the expression of CD117/c-Kit is expedient to enrich for leukemia stem cells (LSCs)\cite{Dick2005} from the leukemic bone marrow\cite{Krivtsov2006,Somervaille2006}. Such LSC-enriched c-Kit+ fractions of independently established MLL-AF9 leukemia were investigated in the present study. 

We chose the MLL-AF9 mouse model system to assess the involvement of enhancers in the leukemogenesis of this subtype of acute myeloid leukemia due to the strong importance of epigenetic irregularities. Mutations within cis-regulatory sequences or in proteins regulating enhancer function epigenetically have emerged as relevant constituents of pathogenic processes\cite{Smith2014}. Enhancer abnormalities are implicated in cancer\cite{Sur2016} and in leukemogenesis in particular\cite{Corces2016}: The mutation\cite{Mansour2014}, deletion\cite{Will2015} or genomic rearrangement\cite{Yamazaki2014,Groeschel2014} of enhancers can result in preleukemic states or full-blown leukemia, if it impacts the expression of known hematopoietic regulators like transcription factors. 

\section{The role of MLL}

The discovery of a frequent pathogenic chromosomal translocation in chronic myeloid leukemia (the Philadelphia chromosome\cite{Chandra2011}) in 1960 sparked a broad search for similar recurrent chromosomal breakpoints in other hematologic neoplasms. Several studies soon identified a fragment on the q-arm of chromosome 11, which was frequently altered in childhood acute leukemia - the locus 11q23, which harbors the \genenamehuman{Kmt2a} gene\cite{Ziemin-vanderPoel1991,Thirman1993,Balgobind2009}. \genenamehuman{Kmt2a} rearrangements are found in roughly 40\% of the infant (0-3 years of age) cases of acute myeloid leukemia, but constitute to less then 5\% of the cases within the in the AYA group (adolescents and young adults, 15–39 years old) and are even more scarce in elderly patients\cite{Bolouri2017}. From a clinical perspective, such \genenamehuman{Kmt2a} rearrangements are strongly associated with monocytic or myelomonocytic phenotypes \cite{Sorensen1994}. 

Although \genenamehuman{Kmt2a} is nowadays the official gene symbol, the name \proteinname{mixed-lineage leukemia protein} (\genenamehuman{MLL1} or \genenamehuman{MLL}) is still widely-used in the scientific literature, while other legacy names like \genenamehuman{HRTX}, \genenamehuman{HRX} or \genenamehuman{ALL-1} have mostly disappeared.

Due to its high similarity, MLL is considered to be the vertebrate homologue of the \emphspecies{Drosophila} positional identity regulator \proteinnamedrosophila{Trithorax}\cite{Gu1992,Tkachuk1992,Djabali1992}

various breakpoints inside MLL\cite{Kobayashi1993} and fusion to various other genes \cite{VanderBurg1999} 


{\emph{Acute leukemia of ambiguous lineage}
\emph{Mixed phenotype acute leukemia with t(v;11q23.3); KMT2A rearranged}\cite{Arber2016}}

MLL-AF6\cite{Numata2018}


Pu.1 also implicated in MLL AML\cite{Aikawa2015}

\section{MLL-AF9}
First reports of a MLL-AF9 fusion were reported as additional recombinations in chronic myelogenous leukemia (CML) \cite{Rowley1973}

The current WHO classification: \emph{AML with t(9;11)(p21.3;q23.3);MLLT3-KMT2A} as one of the eleven subcategories of \emph{AML with recurrent genetic abnormalities}\cite{Arber2016}



therapy induced MLL rearrangement after Topoisomerase inhibitor treatment\cite{Super1993}

Various breakpoints for MLL-AF9 fusions\cite{Super1997}

Approximately, 25\% of FACS-sorted \kithi cells display LSC potential, whereas the \kitlow population has a 14-fold lower frequency of LSCs and contains terminally differentiated leukemic neutrophils\cite{Somervaille2009}





Despite recent progress in diagnosis and leukemogenesis based on genomic landscapes in acute myelogenous leukemia (AML), advances in AML treatment lag behind. Over the past four decades, combination chemotherapy with anthracycline and cytarabine remains the standard induction therapy. Subsequent post-remission consolidation therapy stratifies patients into favorable-risk, intermediate-risk, and unfavorable-risk groups to assign post-remission therapies based on cytogenetic abnormalities and molecular mutations. Allogeneic stem-cell transplant decreases the risk of AML recurrence compared to standard chemotherapy, but it also raises the risk of serious complications. Recent large collections of matched genomic and clinical data may assist in selecting the best individualized therapy for each AML patient. Emerging evidence indicates that molecularly targeted therapies such as FLT3 and IDH inhibitors may be effective in distinct AML subtypes, providing further rationale for a personalized medicine approach. An umbrella trial, such as "BEAT AML Master Trial," designed to offer novel targeted therapy to AML patients based on their genetic characteristics, will be launching worldwide in the near future.




Therapy \cite{Cruickshank2017}
