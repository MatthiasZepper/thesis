

\cite{Xue2019} Mll recognition , also see summary in "A key to unlocking chromatin revealed by complex structures.pdf"


\section{Notable genes}
\label{chap:r:degenes:notable}

\cite{Huang2014} Polycomb enhancers sustain MLL

\paragraph{Malat-1}
lincRNA gene MALAT-1, also known as NEAT2, is part of the long-sought polyadenylated component of nuclear speckles.
its expression is supported by an common enhancer at chr19:5803513-5803789 from the overrepresented clade I.7. (it contains 48 CpGs and has a overall CG content of 68\%). By CAGE-seq it is the transcript with the highest average expression of 17.2\myexp{5} CPM.
see https://elifesciences.org/articles/03058v1 for lots of references including k.o studies of Malat-1.

also see \cite{Long2017} for review on lncRNAs. lncRNA recruits MLL to Hox gene cluster \cite{Yang2014}


\paragraph{Ruvbl1 and Ruvbl2}
general review\cite{Mao2017},
\cite{Breig2014}

Ruvbl2 expression increased upon transformation with MLL-AF9 and is required for for proliferation and survival. \cite{Osaki2013} Furthermore, inhibition of RUVBL2 expression in THP-1 cells led to reduced telomerase activity and clonogenic potential \cite{Osaki2013}. 

In accordance with this finding, we see Ruvbl1 and Ruvbl2 increased in ckit high vs ckit low cells, but not significantly different between +/+ and -c/chip cells
\paragraph{c-Myb}
inhibition of c-myb by degradation\cite{Walf-Vorderwuelbecke2017} (many more cool references)
motifs and cofactors \cite{Bengtsen2015}
\cite{Slany2005,Hess2006}
\paragraph{Hoxa9}
myeloid differentiation block\cite{Fujino2001}, in childhood acute leukemia significant relation to survival and prognosis \cite{Adamaki2015}

ALso relevant for other MLL rearranged leukemia like MLL-ENL \cite{Zeisig2004}

Can be inhibited by phosphorylation through Protein-Kinase C \cite{Vijapurkar2004}
lncRNA recruits MLL to Hox gene cluster \cite{Yang2014}

Hotair and Twist in prostate cancer for HoxA9 activation\cite{Malek2017}

\paragraph{Pum1 und Pum2,FOXP1}
\cite{Spassov2003,Naudin2017}

\paragraph{C/EBPa}
supressed by menin and EZH2 \cite{Thiel2013} in MLL-AF9 leukemia

\paragraph{Jmjd1c}
Highly expressed, targeted by three enhancers (chr10:66645450-66645639
chr10:66576167-66576455 chr10:66589097-66589347), shown to be relevant for MLL-AF9--and HOXA9-mediated acute myeloid leukemia stem cell self-renewal\cite{Zhu2016a} 

\paragraph{Whsc1l1}
Whsc1l1, also known as NSD3 \cite{Shen2015} is targeted by upstream enhancer chr8:26711896-26712138, which is in the highly overrepresented clade Lymphoid Progenitors 8.   \cite{Boettcher2019}

Here, we report the fragment-based discovery of BI-9321, a potent, selective and cellular active antagonist of the NSD3-PWWP1 domain. The human NSD3 protein is encoded by the WHSC1L1 gene located in the 8p11-p12 amplicon, frequently amplified in breast and squamous lung cancer. Recently, it was demonstrated that the PWWP1 domain of NSD3 is required for the viability of acute myeloid leukemia cells. To further elucidate the relevance of NSD3 in cancer biology, we developed a chemical probe, BI-9321, targeting the methyl-lysine binding site of the PWWP1 domain with sub-micromolar in vitro activity and cellular target engagement at 1 µM. As a single agent, BI-9321 downregulates Myc messenger RNA expression and reduces proliferation in MOLM-13 cells. This first-in-class chemical probe BI-9321, together with the negative control BI-9466, will greatly facilitate the elucidation of the underexplored biological function of PWWP domains.


\paragraph{Syk}
"Meis1 increased Syk protein expression and activity. Syk upregulation occurs through a Meis1-dependent feedback loop. By dissecting this loop, we show that Syk is a direct target of miR-146a, whose expression is indirectly regulated by Meis1 through the transcription factor PU.1. In the context of Hoxa9 overexpression, Syk signaling induces Meis1, recapitulating several leukemogenic features of Hoxa9/Meis1-driven leukemia. \cite{Mohr2017}"

Driven by two enhancers, one Myeloid Progenitor and one LymphoidProgenitor \cite{Mohr2017}

targeted by entospletinib (Gilead Sciences) - study (NCT02343939) ongoing: Entospletinib + daunorubicin + cytarabine (Group A), Entospletinib + decitabine (Group B), Entospletinib monotherapy (Group C).


\paragraph{E3 ubiquitin-protein ligases}
Example for E3 ligase with therapeutic potential, but not particularly relevant in our data: X-linked inhibitor of apoptosis protein (XIAP), which targets Caspases for protesomal degradation and hence exerts an antiapopotic function in acute myeloid leukemia\cite{Carter2003,Carter2005} and other malignancies. 

\paragraph{IDH-1}
Frequently mutated in AML, 

targeted by Ivosidenib 

also see IDH-2, which is directly regulated by Dnmt1
in advanced mutant IDH2 patients , enasidenib\cite{Stone2017}
\paragraph{Kat2a}
AML target genes, one of them is Kat2a (also known as Gcn5)\cite{Tzelepis2016}, part of the ATAC complex, a complex with histone acetyltransferase activity on histones H3 and H4, but also acetylates non-histone proteins, such as \tfcebpb\cite{Wiper-Bergeron2007}. While we could not identify relevant enhancers for Kat2a (which is expressed), but the expression of the closely related Kat6b is sustained by two enhancers chr14:22319601-22319937 and chr14:22320305-22320621, which are both part of the overrepresented clade IV.9 (Erytroid progenitors) 

\paragraph{BAX}

Important in MLL-AF4 induced ALL, but also MLL-AF9 rearranges blasts\cite{Benito2015}. Targeted by the overexpressed enhancer chr7:52963044-52963274 (I.3), an interaction only mappable by HiC, as it lies in the intron 2 inside the Dbp gene, which is not expressed in MLL-AF9. The interaction spans 250kb and more than 20 other genes. 

\paragraph{Son of sevenless homolog 2}
The strongest enhancer transcription identified is an enhancer in the first intron of the Guanine-nucleotide releasing factor \genename{Son of sevenless homolog 2}. % Menin interacts with Son of Sevenless and many others \cite{Matkar2013}


\paragraph{Huwe1}
N-myc degradation, ubiquitin ligase\cite{King2016}