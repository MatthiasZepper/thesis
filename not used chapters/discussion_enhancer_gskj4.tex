




\chapter{Enhancer importance}
\minitoc

% % % wörtlich aus Bresnick et al % % %
Intensive scrutiny of human genomes has unveiled considerable genetic variation in coding and noncoding regions. In cancers, including those of the hematopoietic system, genomic instability amplifies the complexity and functional consequences of variation. Although elucidating how variation impacts the protein-coding sequence is highly tractable, deciphering the functional consequences of variation in noncoding regions (genome reading), including potential transcriptional-regulatory sequences, remains challenging. A crux of this problem is the sheer abundance of gene-regulatory sequence motifs (  elements) mediating protein-DNA interactions that are intermixed in the genome with thousands of look-alike sequences lacking the capacity to mediate functional interactions with proteins in vivo. Furthermore, transcriptional enhancers harbor clustered   elements, and how altering a single   element within a cluster impacts enhancer function is unpredictable. Strategies to discover functional enhancers have been innovated, and human genetics can provide vital clues to achieve this goal. Germline or acquired mutations in functionally critical (essential) enhancers, for example at the   locus encoding a master regulator of hematopoiesis, have been linked to human pathologies. Given the human interindividual genetic variation and complex genetic landscapes of hematologic malignancies, enhancer corruption, creation, and expropriation by new genes may not be exceedingly rare mechanisms underlying disease predisposition and etiology. Paradigms arising from dissecting essential enhancer mechanisms can guide genome-reading strategies to advance fundamental knowledge and precision medicine applications. In this review, we provide our perspective of general principles governing the function of blood disease-linked enhancers and  -centric mechanisms.
\cite{Bresnick2019}


% % % wörtlich aus Su et al % % %
These data highly suggest a putative role of H3K27me3 demethylase UTX (KDM6A) in the derepression of these lncRNAs in breast cancer cells, through a mechanism identical to that of coding genes. We then considered whether the H3K27me3 demethylase UTX or the polycomb complex genes (EED, SUZ12, and EZH2) were associated with the patient’s outcome. Strikingly, only SUZ12 and KDM6A were associated with a poor outcome; EZH2 was not.\cite{Su2014a} % % % %



GSK-J1 (and its prodrug form GSK-J4) inhibit Kdm6a and Kdm6b effectively\cite{Kruidenier2012}, but may also show activity against the Kdm5 family\cite{Heinemann2014} and hence inhibit demethylation of not only of H3K27me3/me2 (KDM6), but also of H3K4me3/me2 (KDM5). It has emerged as potential inhibitor for prostate cancer\cite{Morozov2017} and various hematological malignancies \cite{Mathur2017,Boila2017} 


As the H3K4-specific demethylase KDM5B (Jarid1b) has been shown to impair leukemogenesis in murine and human MLL-rearranged AML cells \cite{Wong2015}, GSK-J1 is probably unsuitable. GSK-J4 treatment of bone marrow progenitor cells might by inhibition of Kdm6b lead to activation of hepatic transcription factors \cite{Kochat2017}

IOX1 can also inhibit UTX, but has a higher affinity for Kdm6b/JMJD3, Kdm3a/ JMJD1A, Kdm4a/ JMJD2A, Kdm4e JMJD2E, Kdm4c JMJD2C.\cite{King2010,Schiller2014} Furthermore as a broad spectrum 2-oxoglutarate oxygenase inhibitor \cite{Hopkinson2013}, it will likely activate hypoxic signaling, which itself may impact leukemia growth and self-renewal \cite{Velasco-Hernandez2014,Roychoudhury2015}. 


C/EBP$\alpha$\cite{Rapino2013,Oevelen2015,Cirovic2017}, CEBPA mutations can prime HSCs to generate downstream myeloid leukemia stem cells\cite{Bereshchenko2009} and mutations are crucial perturbations which can lead to AML\cite{Pabst2007,Koschmieder2008}.

C/EBP$\alpha$ in AML
\cite{Hackanson2008}

% wörtlich von Tim C. Somervaille
CEBPA (CCAAT/enhancer binding protein alpha) is a leucine zipper transcription factor (expressed as two isoforms of either 30 kD or 42 kD) that regulates myeloid differentiation in hematopoiesis, as well as the differentiation of other tissue-specific cell types in liver, fat, and lung. It functions in part through repression of E2F and MYC activity, and its complete absence in hematopoietic cells induces pleiotropic effects, including a block in myeloid differentiation accompanied by profound reductions in mature myeloid cell populations, an accumulation of myeloblasts in the bone marrow without progression to AML, and an enhanced self-renewal capacity of HSCs \cite{Somervaille2009a} %%%%%%


Deletion of KDM6A, a histone demethylase interacting with MLL2, in three patients with Kabuki syndrome
D Lederer, B Grisart, MC Digilio, V Benoit… - The American Journal of …, 2012 - Elsevier

The Targeted Disruption of Kdm6a/Utx in Mice Has Gender-Dependent Effects on Primitive Hematopoietic Cell Populations and Myeloid Progenitors 
Ling Tian and Lukas D. Wartman

The Role of Kdm6a in Malignant Hematopoiesis
Ling Tian and Lukas D Wartman