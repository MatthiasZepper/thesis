\chapter{Chromatin domain stability}
\minitoc

% https://www.nature.com/collections/rsxlmsyslk  # Nature review collection !!!!!!!!!!!!!!!!!!!!!!

chromosome domains \cite{Sexton2015}


%% key points of \ cite{Ciabrelli2015} 
Metazoan genomes are highly organized inside the cell nucleus.
TADs do not explain alone their spatial organization.
The chromatin shapes TAD's structure and drives its nuclear positioning and function.
Genome-wide association studies revealed four chromatin types.
Main three-dimensional features of each chromatin type, their relationships with TAD organization and epigenetic memory are described.
%%%%

\section{Heterochromatin formation and propagation}
High compacted DNA, which precludes transcription and replication, but also recombination. In a way heterochromatin can hence be regarded as protective to impede mutagenesis. 

constitutive and facultative types of heterochromatin

Faithful transmission of chromatin domains across replications and the reestablishment of 


Action at a distance: epigenetic silencing of large chromosomal regions in carcinogenesis \cite{Clark2007}

\section{Effects of demethylation on chromatin stability}

Hallo Frank,


Anbei die Erklärung, wie ich auf die Hypothese gekommen bin und was ich heute noch an Papern dazu gesucht/gefunden habe. 


Ausgangspunkt war Cruickshanks et al., die gezeigt hatten, dass die beschriebene Hypomethylierung in seneszenten Fibroblasten der Unterbrechung des Zellzyklus vorausgeht und weiterhin bestehen bleibt, wenn die Seneszenz mittels SV40-T-Antigen überwunden wird. Das warf zwei Fragen auf:

Auch proliferierende Fibroblasten haben bereits PMDs (mehrfach beschrieben und auch im Profil sichtbar). Die Existenz von PMDs allein ist also nicht hinreichend für Seneszenz. Das lasst zwei Interpretationen zu: Es handelt sich dabei um einen reinen Nebeneffekt der keinen Bezug zur Seneszenz hat oder es handelt sich um einen Dosis-Effekt, d.h. leichte PMDs sind kein Problem, wenn sie zu stark hypomethylieren losen sie Seneszenz aus. Bei den Papern, die Irina angehngt hatte, wird durch Dnmt1 uberexpression (Lin et al.) in late-passage MSCs deren Proliferation gesteigert (nicht jedoch durch Dnmt1 mit mutierter katalytischer Domane), was ein Mechanismus über DNA-Methylierung nahe legt. 

Wenn sich die stark hypomethylierten Bereiche in einem Seneszenz-Bypass Methylom nicht zwingend zurückbilden, muss die Zelle den Verlust anderweitig ausgleichen können. In diesem Zusammenhang sei auf das Paper von Lowe et al. verwiesen, die eine gewisse Rückbildung der Methylomveränderungen bei Seneszenz-Exit durchaus gesehen haben wollen (haben aber leider nicht speziell PMDs angeschaut), trotzdem muss es bei Richtigkeit der Hypothese einen weiteren Parameter geben, der beeinflusst, ab wann die Hypomethylierung kritisch wird. Ich halte diesen Parameter für die Zusammensetzung der Lamina. Es ist bekannt, dass sich diese bei Seneszenz, aber auch im Rahmen der normalen Differenzierung von Zellen ändert und mit ihr die Position und Größe von LADs. (Tendenziell werden sie in differenzierten Zellen größer) Leider habe ich noch kein Paper gefunden, dass sich den Seneszenz-Bypass mit Bezug auf Lamina-Veränderungen anschaut. Das würde die Hypothese sehr stützen, wenn man zeigen könnte, dass die Lamina in Seneszenz-Bypass Zellen und Tumoren mit stark ausgeprägten PMDs / hypomethylated Blocks einerseits und andererseits von Zellen, in denen keine oder kaum PMDs beschrieben wurden (weil sie sie nicht tolerieren können) unterscheiden würde.

Dass dieses Ablösen zu Seneszenz führen kann bzw. einer der Mechanismen dabei ist, sieht man im Fall von LaminB1 und dessen Reduktion in der Lamina (Sadaie et al., Shah et al.). Ich würde mich aber nicht rein auf Lamin B1 einschießen, schon deshalb weil laut unseren RNA-seq Daten eigentlich nur Lamb3 (Lamin beta 3) und Lamc1 (Lamin gamma 1) durchgehend exprimiert sind (in WT leicht höher als c/chip) In manchen Samples ist auch noch Lama5 (Lamin alpha 5) schwach exprimiert. Wenn ich die Microarrays hernehme die nach der Barcode-Methode ausgewertet sind, gilt kein einziges Lamin Probeset als exprimiert. 

Soviel im Hinblick auf die Hypomethylierung von LADs mit Seneszenzbezug. Die Idee, dass es sich um eine Art Zusammenbrechen von LADs handeln könnte, ergab sich aus dem Paper Chandra et al. (schon angehängt an Irinas Mail) und dem Paper Kind et al. Die Publikation hat gezeigt, dass besonders in den AT reichen Sequenzen (also PMDs Verdacht) ein Zusammenbrechen von TAD-Grenzen bei Seneszenz stattfindet und auch eine entsprechende Chromatin-Reorganisation einsetzt. Kind et al. erwähnt in die white streaks also mehrere Megabasen lange Bereiche ohne jeglichen Kontakt zur Lamina und zeigt berechnet die Assoziation benachbarter Regionen zu gleichartigem Verhalten. Diese Daten zeigen auf Einzelzellebene, dass ein koordiniertes Ablösen von der Lamina über größe Bereiche stattfinden kann.

An dieser Stelle geht es nur um Heterochromatin, aber dass das in Gedanken fast durch Lamina-Assoziation ersetzt werden kann, ergibt sich aus den Papern Brero et al. und Linhoff et al., die die wichtige Rolle der Methylcytosin-bindenen Proteine MeCP2 und MBD2 bei der Chromatin-Organisation zeigen. Beide sind in unseren LSCs exprimiert. Laut den Arraydaten ist es insbesondere MBD2, das in den LSCs deutlich hochreguliert ist im Vergleich zu den HSCs und das ausgehend von einer bereits stattlichen basalen Expression. Unterschiede zwischen den LSCs gibt es dagegen kaum. Ein anderes Paper Guarda et al. zeigt die Interaktion von MeCP2 mit dem LaminB Rezeptor. 


Zu der Frage, ob die DNA-Methylierung einen Einfluss auf die Bindung an die Lamina hat, habe ich heute noch Paper gesucht. Bereits in einem 20 Jahre alten Review zur replikativen Seneszenz wird diese Vermutung aufgestellt: If 5'-methylCpG modification in mammalian cells serves in part

to optimize the long-term stability of heterochromatin domains, then genomic hypomethylation caused by transient azacytidine treatment should not acutely disrupt heterochromatic regions and, therefore, should not result in immediate growth arrest. Contrariwise, over many cell doublings, the reduction in heterochromatin stability associated with loss of DNA methylation would be predicted to result in a somewhat

shortened proliferative life span, as is observed. (Howard)





Zu guter Letzt kannst du noch einen Blick auf das Paper Espada et al. werfen, insbesondere auf Figure 2A und E. Einerseits haben sie 5mC in Zellen gefärbt damit gezeigt, dass sich eine Chromatin-Ordnung ergibt, bei der 5mC insbesondere inchromocentern organisiert, die mit dem von Linhoff et al. und Brero et al. beschriebenen Pericentromeric Heterochromatin identisch sein dürften. Außerdem zeigt Figure 2A, dass zwei Zelltypen in der identischen Maus Zmpste24 KO-Maus in Folge der Anreicherung von Prolamin A eine völlig abweichende chromosomale Organisation einnehmen können. In beiden Zelltypen normalisiert und gleicht sich die Organisation allerdings wieder an, sobald ProlaminA reduziert wird (siehe die Zmpste24 -/- und Lmna +/- Maus).

Viele Grüße

Matthias

\section{Replication stress}
Heterochromatin formation is linked to replication stress (reviewed in \cite{Nikolov2016}).




The mouse karyotype (Mus musculus) consists of 20 chromosomal arms (all chromosomes are acrocentric), ranging from 58 to 192 Mb in length (on average, 1,24.6 plus minus 32 Mb).\cite{Duret2008}


Kmt5b, deposits the pericentric heterochromatin-specific histone methylation mark H4K20me3, is highly expressed (886,27 FPKM and sustained by enhancer chr19:3768731-3769107 from overrepresented B-cells(VII) 5	clade. Kmt5b (SMYD3) was linked to a more aggressive phenotype of prostate cancer and targets Cyclin D2\cite{Vieira2015}.


\section{Cellular plasticity}

This finding led us to conclude that the gene-centric view of Dnmt1 insufficiency\cite{Trowbridge2012} or DNMTi\cite{Zhou2017} in AML likely falls short of the real complexity involved as promoter hypomethylation is scarce. In a truly exiting perspective Pujadas and Fineberg\cite{Pujadas2012} have proposed a model, how stochastic noise in eu-/heterochromatin borders and a gain in cellular plasticity might be linked to cancer. 

\cite{Timp2013}

\cite{Pujadas2012,Wainwright2017}


"The transient, unstable nature of the highly plastic CSC phenotype is suggestive of the situation in ES cells where active signalling is required for maintenance of the undifferentiated phenotype. In this sense, these cells can be considered to be sitting on a knife edge; primed to differentiate at any moment."\cite{Obrienball2017}

Cellular Decision Making and Biological Noise: From Microbes to Mammals\cite{Balazsi2011}

\cite{Stoeger2016}

The Genomic Landscape of Position Effects on Protein Expression Level and Noise in Yeast - up to 15 fold difference. Tinats\cite{Chen2016b}



This is relevant because especially when short-lifed progenitors are transformed, they rely on Polycomb-proteins to suppress pathways that will lead to differentiation. For example the Polycomb group (PcG) protein BMI1 is required to ensure repression of tumorsuppressor genes that would otherwise counteract the effect of the oncogenic fusions upon leukemic reprogramming of myeloid progenitors by MLL chimeras.\cite{Yuan2011,Smith2011}.

dissociation of Bmi-1 from the Sin3B locus, resulting in increased Sin3B expression and subsequent entry into cellular senescence\cite{DiMauro2015}.

Myeloid-Biased Hematopoietic Stem Cells via a Histamine-Dependent Feedback Loop\cite{Chen2017c}


Random fluctuations in gene expression lead to wide cell-to-cell differences in RNA and protein counts. Most efforts to understand stochastic gene expression focus on local (intrinisic) fluctuations, which have an exact theoretical representation\cite{Sherman2015}

%%%%%%%%%%%%%%%%%%%%%%%%%%
% wörtlich aus Rafique
chromatin organization and epigenetic dysregulation in cancer
We identify large regions of coordinate down-regulation of gene expression, and other regions of coordinate activation, in breast cancers and show that these regions are linked to tumor subtype. In particular we show that a group of coordinately regulated regions are expressed in luminal, estrogen-receptor positive breast tumors and cell lines. For one of these regions of coordinate gene activation, we show that regional epigenetic regulation is accompanied by visible unfolding of large-scale chromatin structure and a repositioning of the region within the nucleus\cite{Rafique2015}
%%%%%%%%%%%%%%%%%%%%%%%%%%%

%%%%%%%%%%%%%%%%%%%%%%%%%%%%%%
% wörtlich aus Forn et al.
Long Range Epigenetic Silencing (LRES) is a mechanism of gene inactivation that affects multiple contiguous CpG islands and has been described in different human cancer types. 
Unexpectedly, benign adenomas in Apc(Min/+) mice showed overexpression of most of the genes affected by LRES in cancer, which suggests that the repressive remodeling of the region is a late event. Chromatin immunoprecipitation analysis of the transcriptional insulator CTCF in mouse colon cancer cells revealed disrupted chromatin domain boundaries as compared with normal cells. Malignant regression of cancer cells by in vitro differentiation resulted in partial reversion of LRES and gain of CTCF binding. We conclude that genes in LRES regions are plastically regulated in cell differentiation and hyperproliferation, but are constrained to a coordinated repression by abolishing boundaries and the autonomous regulation of chromatin domains in cancer cells\cite{Forn2013}
%%%%%%%%%%%%%%%%%%%%%%%%%%%%%%%%

Such LRES may be in part mediated by LINE-1 antisense transription\cite{Cruickshanks2013a} %also mind LINE1, Zpf281 and Mll2 paper


\section{LSD1}

 removes monomethyl and dimethyl groups from histone H3 lysine 4 (H3K4)\cite{Shi2004}, from H3K9 Me1 \& Me2, as well as non-histone targets such as TP53 and DNMT1.
 
 High level expression of KDM1A protein is observed in multiple cancer types,
 
 While transcriptional regulation is important, especially post-transcriptional regulation of protein levels appear to be relevant. 
 
 and its USP22 deubiquitinase, enhancing proteasomal destruction
 
 GSK3B-USP22-KDM1A regulatory axis 

KDM1A/LSD1\cite{Harris2012}

LSD1 inhibition a therapeutic target in \cite{Cusan2018}

In glioblastoma is has been shown that the kinases GSK3$\beta$ and CK1$\alpha$ phosphorylate two serin residues of KDM1a, which enhances binding with and deubiquitylation by USP22. Hence KDM1A protein levels are stabilized by USP22 in glioblastoma\cite{Zhou2016}. Similarly USP28 can stabilize KDM1A in breast cancer \cite{Wu2013} 

Elimination of LSD1 promotes embryonic stem cell (ESC) differentiation toward neural lineage \cite{Han2014} 


Various inhibitors are being developed \cite{Mould2017,Schulz-Fincke2018}
Structure-activity studies on N-Substituted tranylcypromine derivatives lead to selective inhibitors of lysine specific demethylase 1 (LSD1) and potent inducers of leukemic cell differentiation\cite{Schulz-Fincke2018}
therapeutic targeting reviewed\cite{Maiques-Diaz2016}

Enhancer decommission by LSD1\cite{Whyte2012}



% % % % % % % % % % abstract of Han2014
Histone H3K4 demethylase LSD1 plays an important role in stem cell biology, especially in the maintenance of the silencing of differentiation genes. However, how the function of LSD1 is regulated and the differentiation genes are derepressed are not understood. Here, we report that elimination of LSD1 promotes embryonic stem cell (ESC) differentiation toward neural lineage. We showed that the destabilization of LSD1 occurs posttranscriptionally via the ubiquitin-proteasome pathway by an E3 ubiquitin ligase, Jade-2. We demonstrated that Jade-2 is a major LSD1 negative regulator during neurogenesis in vitro and in vivo in both mouse developing cerebral cortices and zebra fish embryos. Apparently, Jade-2-mediated degradation of LSD1 acts as an antibraking system and serves as a quick adaptive mechanism for re-establishing epigenetic landscape without more laborious transcriptional regulations. As a potential anticancer strategy, Jade-2-mediated LSD1 degradation could potentially be used in neuroblastoma cells to induce differentiation toward postmitotic neurons.\cite{Han2014}
% % % % % % % %JADE-2 is the top responder upon Mll2 knockdown. 


% % % % % % % % % % % % % % % % % % % % wörtlich aus Wang2009
\hisfourthree
Histone H3 lysine 4 methylation (H3K4me) has been proposed as a critical component in regulating gene expression, epigenetic states, and cellular identities1. The biological meaning of H3K4me is interpreted by conserved modules including plant homeodomain (PHD) fingers that recognize varied H3K4me states. The dysregulation of PHD fingers has been implicated in several human diseases, including cancers and immune or neurological disorders. Here we report that fusing an H3K4-trimethylation (H3K4me3)-binding PHD finger, such as the carboxy-terminal PHD finger of PHF23 or JARID1A (also known as KDM5A or RBBP2), to a common fusion partner nucleoporin-98 (NUP98) as identified in human leukaemias, generated potent oncoproteins that arrested haematopoietic differentiation and induced acute myeloid leukaemia in murine models. In these processes, a PHD finger that specifically recognizes H3K4me3/2 marks was essential for leukaemogenesis. Mutations in PHD fingers that abrogated H3K4me3 binding also abolished leukaemic transformation. NUP98-PHD fusion prevented the differentiation-associated removal of H3K4me3 at many loci encoding lineage-specific transcription factors (Hox(s), Gata3, Meis1, Eya1 and Pbx1), and enforced their active gene transcription in murine haematopoietic stem/progenitor cells. Mechanistically, NUP98-PHD fusions act as 'chromatin boundary factors', dominating over gene silencing to 'lock' developmentally critical loci into an active chromatin state (H3K4me3 with induced histone acetylation), a state that defined leukaemia stem cells. Collectively, our studies represent, to our knowledge, the first report that deregulation of the PHD finger, an 'effector' of specific histone modification, perturbs the epigenetic dynamics on developmentally critical loci, catastrophizes cellular fate decision-making, and even causes oncogenesis during mammalian development.\cite{Wang2009}
% % % % % % % % % % % % % % % % % % % % % % % % % % %55