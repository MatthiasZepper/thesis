\chapter{Senescence}
\minitoc

Ich nehme an, du bist eben im Seminar bei den Stichworten „Seneszenz“, „AML“ und „Dnmt1“ ebenfalls hellhörig geworden. Leider sieht es nicht so aus, als ob (zumindest auf mRNA-Ebene) ein Verlust von Reptin mechanistisch mit der Seneszenz in –/chip-LSCs verknüpft wäre. 

Die Expression von Reptin (Ruvbl2) ist in der WT LSC im Vergleich zur HSC unverändert, in –/chip leicht erhöht. Die Expression von Pontin in der wt LSC sogar niedriger und in der -/chip scheinbar erhöht, möglicherweise in einem Versuch den niedrigen Dnmt1 Level zu kompensieren. 



Dnmt1 is preferably recruited to CGIs.
Other enzymes (Dnmt3a/3b) maintain or methylate de novo the CGIs

Due to the general association of demethylation effect size within regions and their presumable location, constitutive genomic areas in general, we wondered if in turn the measured methylation 

however in some regions we noticed a clear discrepancy between the annotation and the determined methylation. We therefore used our smoothed methylation 

early active genes, GC rich promoters, acquire H3K27me3 when inactive, \cite{Xie2013}, late active genes without developmental relevance are silenced by promoter methylation. 

One thing you might want to note done as a topic for the discussion would be the relationship of PMDs with LADs. Because I just got reminded by your request of the title of the Meuleman Paper and that they explicitly link constitutive LADs with A/T rich sequence. It may be a coincidence, but Gaidatzis let al. inked the location of PMDs also at preferably to A/T rich sequences. 

Since our genome consists of more A/T than C/G there are many A/T rich sequences and this is of course a weak link. but together with the observation that PMDs have mainly be reported in fast dividing cancer cells or cultured cell lines (except for placenta cells, which are of trophoblast origin: http://www.ncbi.nlm.nih.gov/pmc/articles/PMC3955088/ ), the question is appealing, whether the strength of PMDs reflects the amount of senescent cells within a population. This would fit to the observation of Berman et al that in normal colon cell populations the sites of PMDs in colon cancer are already slightly demethylated. 

It may be that in the case of cancer cells overcame the growth limiting barriers usually associated with such a widespread demethylation, which would then be the typical state a senescent cells genome would be in. Ziller et al. reported only 20\% of CpGs dynamic in a normal developmental context, but found many dynamic (that is changing in methylation) more if they also considered cancer or cultured cell linkes (up to ~60\% of total CpGs).

In the nice comment cited above (http://www.ncbi.nlm.nih.gov/pmc/articles/PMC3955088/ - which I linked directly in Pubmed Central since the regular journal version is not accessible) the author asks, at which timepoint PMDs may occur in a regular cells life – maybe in senescence is the correct answer when you puzzle the bits and pieces of information together to a big comprehensive view.

The only problem of the hypothesis PMDs are actually hallmarks of a senecent cell’s genome at late-replicating, A/T rich lamina-associated sites is my problem of accurately identifying PMDs in our c/chip LSCs. Maybe I should also have a second look there…or we just stick with the term PMD-like and just point this out as a possible mechanism. 


% % % % % % % wörtlich aus Bartkova 
Recent studies have indicated the existence of tumorigenesis bar-
riers that slow or inhibit the progression of preneoplastic lesions
to neoplasia. One such barrier involves DNA replication stress,
which leads to activation of the DNA damage checkpoint and
thereby to apoptosis or cell cycle arrest 1,2 , whereas a second barrier
is mediated by oncogene-induced senescence 3–6 . The relationship
between these two barriers, if any, has not been elucidated. Here
we show that oncogene-induced senescence is associated with
signs of DNA replication stress, including prematurely terminated
DNA replication forks and DNA double-strand breaks. Inhibiting
the DNA double-strand break response kinase ataxia telangiecta-
sia mutated (ATM) suppressed the induction of senescence and in
a mouse model led to increased tumour size and invasiveness.
Analysis of human precancerous lesions further indicated that
DNA damage and senescence markers cosegregate closely. Thus,
senescence in human preneoplastic lesions is a manifestation of
oncogene-induced DNA replication stress and, together with
apoptosis, provides a barrier to malignant progression.
There are several forms of senescence. Replicative senescence, the
form induced by eroded telomeres, depends on activation of the
DNA double-strand break (DSB) checkpoint kinases ATM and
checkpoint kinase 2 (Chk2) 7–9 . A second form of senescence induced
by agents that cause DNA DSBs also depends on ATM and Chk2 (ref.
10). By contrast, oncogene-induced senescence, the form of sen-
escence that is most likely to be associated with human precancerous
lesions 3,4 , has been linked to increased expression of the tumour
suppressors p16 INK4A and ARF, rather than to activation of the
DNA DSB checkpoint pathway 11–13\cite{Bartkova2006}
% % % % % % %

pericentric chromatin transcripts can cause senescence -> spurious transcription\cite{Tasselli2016}