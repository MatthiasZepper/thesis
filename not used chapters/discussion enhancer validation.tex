
\section{Enhancer calling and validation in general}
\label{chap:d:enhancers:mechanism:validate}

Throughout the entire thesis, the CAGE-defined enhancers are consistently referred to as putative or presumable enhancers to emphasize the lack of experimental validation. This choice has been made on grounds of the imprecision inherent in any sequencing experiment and particularly in those addressing cis-regulatory elements like enhancers. The promise made by a genome-wide technique is, to get the full, unbiased picture and thus the ultimate answer to the question raised. However, even putting mappability or technical errors aside, this promise can hardly be kept, even though numbers like the claim to have identified exactly \num{48415} enhancers in murine hematopoiesis\cite{Lara-Astiaso2014} might suggest otherwise. 

All techniques \dissrefpage{chap:i:enhancers:methods} to identify enhancers will preferably pick up elements in a specific state \dissrefpage{chap:i:enhancers:states} or rely on features not exclusive to enhancers. Anyway, the balance between sensitivity and specificity needs to be struck. Furthermore, if multiple methods are combined for reciprocal validation, the result will suffer from the multiplication of error sources \citerev{Halfon2018}. 

Even reporter assays, often considered as the gold-standard of enhancer biology, may fail, since the ability of a fragment to act as an enhancer also depends on its chromosomal context and the transcription factors expressed in a cell. Furthermore, many enhancers consist of clustered elements, which only take effect when they are combined. This is illustrated by the cis-regulatory-elements upstream of \genenamehuman{SPI1}-locus (\proteinnamehuman{PU.1}), which comprise proven enhancers, yet will not work in isolated luciferase assays. 

Another challenge is the sheer abundance of look-alike gene-regulatory elements, which lack the capacity to mediate functional interactions in vivo, but might obfuscate the true enhancers in assays\citerev{Bresnick2019}. 

Therefore, no method can claim to have exactly identified all enhancers of a particular cell type as exemplified by a recent comparative study\cite{Benton2019}. Even the quite reliable method based on CAGE-seq data may have a particular bias towards other cis-regulatory elements\cite{Young2017}. Yet, in the original publication on which we based our approach, roughly \SI{70}{\percent} of the randomly selected loci for validation in enhancer reporter assays showed significant activity compared to control\cite{Andersson2014}, whereas \SI{30}{\percent} or less of the predicted enhancers from chromatin data were positively validated in other studies\cite{Kheradpour2013,Kwasnieski2014}. 

In addition to the superior validation rate of enhancers called from loci with balanced bidirectional (divergent) transcription\cite{Andersson2014,Nagari2017}, our the clustering approach allowed for additional validation: Evidently, the likelihood that a false positive enhancer reflected a known chromatin signature across the whole hematopoietic tree was very low. On the downside, the authenticity of entirely leukemia-specific enhancers might be underestimated by the approach, since they were not backed by healthy chromatin signatures accordingly.